\documentclass{article}

\usepackage{ctex}
\usepackage{tikz}
\usetikzlibrary{cd}
\usetikzlibrary{decorations.pathreplacing}

\usepackage{amsthm}
\usepackage{amsmath}
\usepackage{amssymb}

\usepackage{unicode-math}

\usepackage{enumitem}

\usepackage[textwidth=18cm]{geometry} % 设置页宽=18

\usepackage{blindtext}
\usepackage{bm}
\parindent=0pt
\setlength{\parindent}{2em} 
\usepackage{indentfirst}



\usepackage{xcolor}
\usepackage{titlesec}
\titleformat{\section}[block]{\color{blue}\Large\bfseries\filcenter}{}{1em}{}
\titleformat{\subsection}[hang]{\color{red}\Large\bfseries}{}{0em}{}
%\setcounter{secnumdepth}{1} %section 序号

\newtheorem{theorem}{Theorem}[section]
\newtheorem{lemma}[theorem]{Lemma}
\newtheorem{corollary}[theorem]{Corollary}
\newtheorem{proposition}[theorem]{Proposition}
\newtheorem{example}[theorem]{Example}
\newtheorem{definition}[theorem]{Definition}
\newtheorem{remark}[theorem]{Remark}
\newtheorem{exercise}{Exercise}[section]

\newcommand*{\xfunc}[4]{{#2}\colon{#3}{#1}{#4}}
\newcommand*{\func}[3]{\xfunc{\to}{#1}{#2}{#3}}

\newcommand\Set[2]{\{\,#1\mid#2\,\}} %集合
\newcommand\SET[2]{\Set{#1}{\text{#2}}} %

\begin{document}
\title{Fourier Analysis}
\author{枫聆}
\maketitle
\tableofcontents

\section{The Genesis of Fourier Analysis}
%wave equation

\newpage
\section{傅里叶级数的基本性质}

\subsection{欧拉公式}
频繁的出现,总会忘记它的堆导,不如一开始就记录下来.
%https://www.zhihu.com/question/41134540
欧拉最早是通过$e^x,\sin x ,\cos x$的泰勒展开式观察出来的欧拉公式
\begin{align*}
e^ x =1+x+\frac{1}{2!}x^2+\frac{1}{3!}x^3+\cdots \\
sin(x)=x-\frac{1}{3!}x^3+\frac{1}{5!}x^5+\cdots \\
cos(x)=1-\frac{1}{2!}x^2+\frac{1}{4!}x^4+\cdots
\end{align*}

把$x=i\theta$带入$e^x$的泰勒展开式.

\begin{equation}
\begin{aligned}
e^{i\theta } & = 1 + i\theta + \frac{(i\theta )^2}{2!} + \frac{(i\theta )^3}{3!} + \frac{(i\theta )^4}{4!} + \frac{(i\theta )^5}{5!} + \frac{(i\theta )^6}{6!} + \frac{(i\theta )^7}{7!} + \frac{(i\theta )^8}{8!} + \cdots \\ 
& = 1 + i\theta - \frac{\theta ^2}{2!} - \frac{i\theta ^3}{3!} + \frac{\theta ^4}{4!} + \frac{i\theta ^5}{5!} - \frac{\theta ^6}{6!} - \frac{i\theta ^7}{7!} + \frac{\theta ^8}{8!} + \cdots \\ 
& = \left( 1 - \frac{\theta ^2}{2!} + \frac{\theta ^4}{4!} - \frac{\theta ^6}{6!} + \frac{\theta ^8}{8!} - \cdots \right) + i\left(\theta -\frac{\theta ^3}{3!} + \frac{\theta ^5}{5!} - \frac{\theta ^7}{7!} + \cdots \right) \\ 
& = \cos \theta + i\sin \theta
\end{aligned}
\end{equation}

简单而优雅,$e^{i\theta}$是一个圆周运动.


\subsection{黎曼可积}

\begin{definition}
\rm 定义在$[O,L]$上实数函数$f$,如果满足$f(x)$有界,且对任意的$\varepsilon > 0$,存在一个$[0,L]$上子划分$0 = x_0 < x_1 < \cdots < x_{N-1} < x_N = L$,让$\mathcal{U}$和$\mathcal{L}$分别表示在这个子划分上的upper and lower sums of $f$. \[\mathcal{U} = \sum\limits_{j=1}^{N} \left[ \sup\limits_{x_{j-1} \leq x \leq x_{j}}f(x) \right] (x_j-x_{j-1}). \]and\[\mathcal{L} = \sum\limits_{j=1}^{N} \left[ \inf\limits_{x_{j-1} \leq x \leq x_{j}}f(x) \right] (x_j-x_{j-1}).\]使得$\mathcal{U} - \mathbb{L} < \varepsilon.$,则称这个函数$f$黎曼可积(Riemann integrable).
\end{definition}

直觉上只要如果$f$黎曼可积,只要划分的足够细,总能满足上述条件.

\begin{example}
\rm 定义在$[0,1]$上函数.
\begin{equation}
f(x)=\left\{
\begin{aligned}
1  &\qquad\frac{1}{n+1} < x \leq \frac{1}{n},\ n\ \text{is odd}, \\
0  &\qquad\frac{1}{n+1} < x \leq \frac{1}{n},\ n\ \text{is even}, \\
0  &\qquad x=0.
\end{aligned}
\right.
\end{equation}
虽然这个$f$在$x=\frac{1}{n}$和$0$上并不是连续的,但是这个$f$是黎曼可积. 只要划分足够细,覆盖$\frac{1}{n}$和$0$的子区间就足够小,对应的upper sum和lower sum的差值就越小.
\end{example}



\end{document}