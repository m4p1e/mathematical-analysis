\documentclass{article}

\usepackage{ctex}
\usepackage{tikz}
\usetikzlibrary{cd}
\usetikzlibrary{decorations.pathreplacing}

\usepackage{amsthm}
\usepackage{amsmath}
\usepackage{amssymb}

\usepackage{unicode-math}

\usepackage{enumitem}

\usepackage[textwidth=18cm]{geometry} % 设置页宽=18

\usepackage{blindtext}
\usepackage{bm}
\parindent=0pt
\setlength{\parindent}{2em} 
\usepackage{indentfirst}



\usepackage{xcolor}
\usepackage{titlesec}
\titleformat{\section}[block]{\color{blue}\Large\bfseries\filcenter}{}{1em}{}
\titleformat{\subsection}[hang]{\color{red}\Large\bfseries}{}{0em}{}
%\setcounter{secnumdepth}{1} %section 序号

\newtheorem{theorem}{Theorem}[section]
\newtheorem{lemma}[theorem]{Lemma}
\newtheorem{corollary}[theorem]{Corollary}
\newtheorem{proposition}[theorem]{Proposition}
\newtheorem{example}[theorem]{Example}
\newtheorem{definition}[theorem]{Definition}
\newtheorem{remark}[theorem]{Remark}
\newtheorem{exercise}{Exercise}[section]

\newcommand*{\xfunc}[4]{{#2}\colon{#3}{#1}{#4}}
\newcommand*{\func}[3]{\xfunc{\to}{#1}{#2}{#3}}

\newcommand\Set[2]{\{\,#1\mid#2\,\}} %集合
\newcommand\SET[2]{\Set{#1}{\text{#2}}} %

\begin{document}
\title{Fourier Analysis}
\author{枫聆}
\maketitle
\tableofcontents

\newpage
\section{The Genesis of Fourier Analysis}
%wave equation
\section{motivation}

我们来问一个非常基础的问题: 给定$[0,\pi]$上的函数$f$($f(0) = f(\pi)$),我们能否找到系数$A_m$,使得下面等式成立?

$$
f(x) = \sum\limits_{m=1}^\infty A_m\sin mx
$$

刚遇到这个问题,你会发现有点莫名奇妙的意思. 可能会想为什么会问这个问题?其实我现在也不知道... 等学的过程中发现了再告诉你. 回到正题,这个问题是后面学习fourier analysis的比较重要的东西. 我们两边同乘$\sin nx$,然后再积分.

$$
\begin{aligned}
\int_0^\pi f(x)\sin nxdx &= \int_0^\pi \left(\sum\limits_{m=1}^\infty A_m \sin mx \right) \sin nx dx \\
&= \sum\limits_{m=1}^\infty A_m \int_0^\pi \sin mx \sin nx dx = A_n \cdot \frac{\pi}{2}.
\end{aligned}
$$

这里用到了一个fact即
$$
\int_0^\pi \sin mx \sin nx dx = \left \{
\begin{array}{lr}
0  & \text{if} \ m \neq m \\
\frac{\pi}{2} & \text{if} \ m = n
\end{array} \right.
$$

这个fact左边可以写成:
$$
\int_0^\pi \frac{\cos (mx-nx) - \cos (mx+nx)}{2}.
$$

在这里$m,n$都非负,所以只需要考虑$m=n$. $cos(mx+nx)$在$[0,\pi]$上积分为$0$. 因此
$$
A_n = \frac{2}{\pi} \int_0^\pi \sin mx \sin nx dx.
$$

$A_n$表示$n^{th}$ Fourier sine coefficient of $f$. 如果$f$刚好是一个奇函数函数则我们可以把定义域从$[0,\pi]$扩展到更一般地$[-\pi,\pi]$上. 类似我们可能会想如果$[-\pi,\pi]$上一个偶函数$g$,它是否可以表示成$\cos$无穷级数呢?

$$
g(x) = \sum\limits_{m=0}^{\infty} A_m' \cos mx.
$$

更一般地,任意的函数$F$在$[-\pi,\pi]$上都可以表示为一个奇函数和一个偶函数的和.

$$
F(x) = \frac{F(x)-F(-x)}{2} +\frac{F(x)+F(x)}{2}.
$$

让$f(x) = \frac{F(x)-F(-x)}{2} $和$g(x) = \frac{F(x)+F(x)}{2}$. 这时候我们可能会问$F$能否写成下面的形式.

$$
F(x) = \sum\limits_{m=1}^\infty A_m\sin mx + \sum\limits_{m=0}^{\infty} A_m' \cos mx.
$$

这里我们再使用欧拉公式简化一下.

$$
F(x) = \sum\limits_{m=-\infty}^\infty a_me^{imx}.
$$

注意这里的负号,这是因为$\sin x = \frac{e^{imx}-e^{-imx}}{2i}$和$\cos x = \frac{e^{imx} + e^{-imx}}{2}$这两个东西在这. 类似地,我可以用前面方法两边同时乘上一个$e^{-inx}$,再积分.

$$
	\int_{-\pi}^{\pi} F(x)e^{-inx}dx = \int_{-\pi}^{\pi} \left( \sum\limits_{m=-\infty}^\infty a_me^{imx} \right)e^{-inx}dx.
$$

这里同样可以用一个fact.
$$
\frac{1}{2\pi} \int_{-\pi}^{\pi} e^{imx}e^{-inx} dx =\left\{ \begin{array}{lc}
0 & \text{if}\ n \neq m \\
1 & \text{if}\ n = m
\end{array} \right.
$$

因此我们可以得到$a_n$.

$$
	a_n = \frac{1}{2\pi} \int_{-\pi}^{\pi} F(x)e^{-inx}dx.
$$

这个$a_n$就是所谓的第$n$个$F$的傅里叶系数. 所以我们的问题来了: 给定一个$[-\pi,\pi]$上一个reasonable function $F(x)$. 然后用上面方法构造一系列系数,那么下面的等式是否成立呢?

$$
F(x) = \sum\limits_{-\infty}^{\infty} a_me^{imx}
$$

Joseph Fourier 是第一个相信“任意”的函数$F$都可以被表示成上述的级数. 也就是说他相信任意的函数都可以表示成$\cos sin mx$和$\cos nx$的线性组合(有可能是无限个). 对此的第一个证明是Dirichlet.

\newpage
\section{傅里叶级数的基本性质}

\subsection{欧拉公式}
频繁的出现,总会忘记它的堆导,不如一开始就记录下来.
%https://www.zhihu.com/question/41134540
欧拉最早是通过$e^x,\sin x ,\cos x$的泰勒展开式观察出来的欧拉公式
\begin{align*}
e^ x =1+x+\frac{1}{2!}x^2+\frac{1}{3!}x^3+\cdots \\
sin(x)=x-\frac{1}{3!}x^3+\frac{1}{5!}x^5+\cdots \\
cos(x)=1-\frac{1}{2!}x^2+\frac{1}{4!}x^4+\cdots
\end{align*}

把$x=i\theta$带入$e^x$的泰勒展开式.

\begin{equation}
\begin{aligned}
e^{i\theta } & = 1 + i\theta + \frac{(i\theta )^2}{2!} + \frac{(i\theta )^3}{3!} + \frac{(i\theta )^4}{4!} + \frac{(i\theta )^5}{5!} + \frac{(i\theta )^6}{6!} + \frac{(i\theta )^7}{7!} + \frac{(i\theta )^8}{8!} + \cdots \\ 
& = 1 + i\theta - \frac{\theta ^2}{2!} - \frac{i\theta ^3}{3!} + \frac{\theta ^4}{4!} + \frac{i\theta ^5}{5!} - \frac{\theta ^6}{6!} - \frac{i\theta ^7}{7!} + \frac{\theta ^8}{8!} + \cdots \\ 
& = \left( 1 - \frac{\theta ^2}{2!} + \frac{\theta ^4}{4!} - \frac{\theta ^6}{6!} + \frac{\theta ^8}{8!} - \cdots \right) + i\left(\theta -\frac{\theta ^3}{3!} + \frac{\theta ^5}{5!} - \frac{\theta ^7}{7!} + \cdots \right) \\ 
& = \cos \theta + i\sin \theta
\end{aligned}
\end{equation}

简单而优雅,$e^{i\theta}$是一个圆周运动.


\subsection{黎曼可积}

\begin{definition}
\rm 定义在$[O,L]$上实数函数$f$,如果满足$f(x)$有界,且对任意的$\varepsilon > 0$,存在一个$[0,L]$上子划分$0 = x_0 < x_1 < \cdots < x_{N-1} < x_N = L$,让$\mathcal{U}$和$\mathcal{L}$分别表示在这个子划分上的upper and lower sums of $f$. \[\mathcal{U} = \sum\limits_{j=1}^{N} \left[ \sup\limits_{x_{j-1} \leq x \leq x_{j}}f(x) \right] (x_j-x_{j-1}). \]and\[\mathcal{L} = \sum\limits_{j=1}^{N} \left[ \inf\limits_{x_{j-1} \leq x \leq x_{j}}f(x) \right] (x_j-x_{j-1}).\]使得$\mathcal{U} - \mathbb{L} < \varepsilon.$,则称这个函数$f$黎曼可积(Riemann integrable).
\end{definition}

直觉上只要如果$f$黎曼可积,只要划分的足够细,总能满足上述条件.

\begin{example}
\rm 定义在$[0,1]$上函数.
\begin{equation}
f(x)=\left\{
\begin{aligned}
1  &\qquad\frac{1}{n+1} < x \leq \frac{1}{n},\ n\ \text{is odd}, \\
0  &\qquad\frac{1}{n+1} < x \leq \frac{1}{n},\ n\ \text{is even}, \\
0  &\qquad x=0.
\end{aligned}
\right.
\end{equation}
虽然这个$f$在$x=\frac{1}{n}$和$0$上并不是连续的,但是这个$f$是黎曼可积. 只要划分足够细,覆盖$\frac{1}{n}$和$0$的子区间就足够小,对应的upper sum和lower sum的差值就越小.
\end{example}

\subsection{Function on the circle}

有一个周期为$2\pi$的周期函数和定义在单位元上的函数之间非常自然的关系式.

$$
f(\theta) = F(e^{i\theta}).
$$

单位圆上的点用$e^{i\theta}$表示. where $\theta$ is a real number
that is unique up to integer multiples of $2\pi$(这句话我不明白up to是什么意思? 可能是说$\theta$和$\theta + 2k\pi$是等价呢?)其中$F$是定义在单位元上的函数. 然后用每个$\delta$作为定义域,构造一个函数$f$,可以看出来$f$是$\mathbb{R}$上周期为$2\pi$的周期函数. 因为$f(\theta) = f(\theta + 2\pi).$

它们两者之间会都会保留比较好的性质,例如如果$f$在长度为$2\pi$区间上可积则$F$也同样在单位圆上可积. 连续,可微也是相同.

\subsection{主要的定义}
开始正式的学习fourier analysis,首先给出fourier series的准确定义. 

\begin{definition}
\rm 给定$f$在长度为$L$的区间$[a,b]$($L=b-a$)上可积. 则$f$的第$n$项的fourier coefficient定义为
$$
	\widehat{f}(n) = \frac{1}{L} \int_{a}^{b} f(x)e^\frac{{-2\pi inx}}{L}dx, \ n \in \mathbb{Z}.
$$

$f$的fourier series表示为
$$
\sum\limits_{n=-\infty}^{\infty} \widehat{f}(n)e^{\frac{2\pi inx}{L}}.
$$
\end{definition}

为什么这里的$L$被放到了系数上?相对于把周期放缩到了$L$. 这样做的目的,我其实还不知道为什么? 难道是为了把$f(x)$作为周期为$L$的周期函数扩展到整个$\mathbb{R}$上?

如果$f$在$[-\pi,\pi]$可积,所对应的fourier coefficient可以写的更加简洁.

$$
	\widehat{f}(n) = a_n = \frac{1}{2\pi} \int_{-\pi}^{\pi} f(\theta)e^{-in\theta}d\theta, \ n \in \mathbb{Z}.
$$

完整的fourier series为.

$$
f(\theta) \sim \sum\limits_{n=-\infty}^{\infty} a_n e^{in\theta}.
$$

\subsection{Convergence of Fourier Series}
 
fourier series 属于三角级数的一种. 如果三角级数只有有限多项,则称为一个三角多项式,其最高次项$n$的为其系数(dergee). $\sum\limits_{-\infty}^{\infty} c_n e^{\frac{2\pi inx}{L}}$是一个标准的fourier series,有限多项就是指当$n > |k|$时$c_n = 0$,这个$|k|$就表示当前的三角多项式的系数.
 
 
\begin{definition}
\rm fourier series的部分和(partial sum)表示为.
$$
S_N(f)(x) = \sum\limits_{n=-N}^{N} \widehat{f}(n) e^{\frac{2\pi inx}{L}}.
$$
其中$N$是一个正整数.
\end{definition}

\begin{example}
\rm 狄利克雷核 $N^{th}$ Dirichlet kernel. $x \in [-\pi,\pi].$

$$
D_N(x) = \sum\limits_{-N}^{N} e^{inx}.
$$ 

注意这里的fourier coefficient $a_n = 1$ if $n \leq N$ and $a_n = 0$ otherwise. 可以推出更简洁的式子,定义$\omega = e^{ix}$. $D_N(x)$可以分解成两个partial sum.
$$
	\sum\limits_{0}^{N} \omega^n\ \text{and}\ \sum\limits_{-N}^{-1} \omega^n 
$$
分别计算这个部分可以得到
$$
	\frac{1-\omega^{N+1}}{1-\omega}\ \text{and}\ \frac{\omega^{-N}-1}{1-\omega}
$$
它们的和为
$$
\frac{\omega^{-N}-\omega^{N+1}}{1-\omega} =  \frac{\omega^{-N-\frac12}-\omega^{N+\frac12}}{\omega^{-\frac12}-\omega^{\frac12}} = \frac{\sin((N+\frac12)x)}{\sin(\frac12x)}.
$$
\end{example}

\begin{example}
\rm 泊松核 Poisson kernel. $\theta \in [-\pi,\pi]\ \text{and}\ 0 \leq r \leq 1.$
$$
P_r(\theta) = \sum\limits_{-\infty}^{\infty} r^{|n|}e^{inx}.
$$
\end{example}

好吧关于fourier series收敛的主题没有那么简单,至少在这里还没什么可记录的...

\subsection{Uniqueness of Fourier series}
假设fourier series在某种合适的sense下收敛,现在再问一个问题,一个函数是否可以被其对应的fourier series唯一确定呢?也就是现在$f$和$g$有相同的fourier series,那么$f$和$g$是不是相同的呢?这个问题有可以转换为如果$\widehat{f}(n)=0$对所有的$n \in \mathbb{Z}$成立,则$f=0$. 这个命题不能直接下结论,因为如果两个函数只在有限多个不同点上函数值不一样,则它们的黎曼积分还是一样的,所以算出来的fourier coefficient都是对应都是一样的. 也就是如果两个只有有限多个点上函数值不同的函数,它们的fourier series是相同的. 当然了也有一些非常positive result.

\begin{theorem}
Suppose that $f$ is an integrable function on the circle with
$fˆ(n)=0$ for all $n \in \mathbb{Z}$. Then $f(θ_0)= 0$ whenever $f$ is continuous at the point $θ_0$.
\end{theorem}
 
\end{document}
%http://math.uchicago.edu/~may/REU2017/REUPapers/Xue.pdf
