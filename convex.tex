\documentclass{article}

\usepackage{ctex}
\usepackage{tikz}
\usetikzlibrary{calc}
\usetikzlibrary{cd}
\usetikzlibrary{decorations.pathreplacing}

\usepackage{amsthm}
\usepackage{amsmath}
\usepackage{amssymb}

\usepackage{hyperref} %url
\hypersetup{
    colorlinks=true,
    linkcolor=blue,
    filecolor=magenta,      
    urlcolor=cyan,
    pdftitle={Overleaf Example},
    pdfpagemode=FullScreen,
    }

\usepackage{enumitem}

\usepackage[textwidth=18cm]{geometry} % 设置页宽=18

\usepackage{blindtext}
\usepackage{bm}
\parindent=0pt
\setlength{\parindent}{2em} 
\usepackage{indentfirst}



\usepackage{xcolor}
\usepackage{titlesec}
\titleformat{\section}[block]{\color{blue}\Large\bfseries\filcenter}{}{1em}{}
\titleformat{\subsection}[hang]{\color{red}\Large\bfseries}{}{0em}{}
%\setcounter{secnumdepth}{1} %section 序号

\newtheorem{theorem}{Theorem}[section]
\newtheorem{lemma}[theorem]{Lemma}
\newtheorem{corollary}[theorem]{Corollary}
\newtheorem{proposition}[theorem]{Proposition}
\newtheorem{example}[theorem]{Example}
\newtheorem{definition}[theorem]{Definition}
\newtheorem{remark}[theorem]{Remark}
\newtheorem{exercise}{Exercise}[section]
\newtheorem{annotation}[theorem]{Annotation}

\newcommand*{\xfunc}[4]{{#2}\colon{#3}{#1}{#4}}
\newcommand*{\func}[3]{\xfunc{\to}{#1}{#2}{#3}}

\newcommand\Set[2]{\{\,#1\mid#2\,\}} %集合
\newcommand\SET[2]{\Set{#1}{\text{#2}}} %

\newcommand{\norm}[1]{\left\lVert#1\right\rVert} % 范数
\newcommand{\vect}[1]{\mathbf{#1}} % vector

\newcommand{\thereal}{\mathbf{R}} %实数 
\newcommand{\inprod}[2]{\left<#1,#2\right>} % inner product

\begin{document}
\title{凸函数的世界 \\ \small{{\color{red}凸分析}和{\color{red}凸优化}}}

\author{枫聆}
\maketitle
\tableofcontents

\newpage
\section{数学优化问题}

%the new world!
\begin{definition}
\rm {\color{red}数学优化}问题或者说{\color{red}优化问题}可以写成如下形式
$$
\begin{array}{ll}
\text{minimize} & f_0(x) \\
\text{subject to} & f_i(x) \leq b_i, \; i = 1,2,\cdots,m.
\end{array}
$$
其中向量$x = (x_1,\cdots,x_n) \in \mathbf{R}^n$称为问题的{\color{red}优化变量},函数$\func{f_0}{\mathbf{R}^n}{\mathbb{R}}$称为{\color{red}目标函数},函数$\func{f_i}{\mathbf{R}^n}{\mathbb{R}}$被称为{\color{red}约束函数},常数$b_i$称为{\color{red}约束上限}或者{\color{red}约束边界}.
\end{definition}

\begin{definition}
\rm  那些满足约束的向量$z$,即使得上述不等式成立的向量,它们构成一个{\color{red}解集} $Z$. 这个解集中使得$f_0(z)$最小的那些$x^*$称为当前优化问题的{\color{red}最优解},即
$$
\forall z \in Z, f_i(z) \leq b_i,\;i=1,2,\cdots,m~\text{and}~f(x^*) \leq f(z).
$$
\end{definition}

\newpage
\section{基本概念}

\subsection{仿射集(affine set)}

\begin{definition}
\rm {\color{red} ($\thereal^n$($n$维实数向量空间)上直线的定义)} 对任意两个$\thereal^n$中不同两个$\mathbf{x}$和$\mathbf{y}$,形如
$$
\mathbf{x} + \lambda(\mathbf{y}-\mathbf{x}) = (1-\lambda)\mathbf{x} + \lambda\mathbf{y},\; \lambda \in \thereal  
$$
的点集被称为过$\mathbf{x}$和$\mathbf{y}$的直线.
\end{definition}

\begin{definition}
\rm 对于$\thereal^n$中的子集$M$,如果对于任意的$\vect{x},\vect{y} \in M$和$\lambda \in \thereal$都有$(1-\lambda)\vect{x} + \lambda\vect{y} \in M$,则称$M$为$\thereal^n$中的{\color{red}仿射集}(affine set). 相关书与仿射集同义的名词有{\color{red}仿射流形
}(affine manifold),{\color{red}仿射变量}(affine variety),{\color{red}线性变量}(linear variety)或者{\color{red}flat}(平坦的).
\end{definition}

\begin{theorem}
\rm $\thereal^n$上的所有子空间都是仿射集. 反过来含$\vect{0}$的仿射集都是子空间.
\end{theorem}

\begin{proof}
\rm 由向量空间子空间的定义,给定任意的子空间$V$,在{\color{red}scalar multiplication}和{\color{red} vector addition}下封闭的,自然也满足仿射集的定义. 

相反,若给定一个仿射集$M$,有$\vect{0} \in M$,因此
$$
\forall \vect{x} \in M, \lambda x = (1-\lambda)\vect{0} + \lambda \vect{x} \in M,
$$
这就证明了{\color{red}scalar multiplication}. 对任意的$\forall \vect{x},\vect{y} \in M$,有
$$
\vect{x}+\vect{y} = 2(\frac{1}{2} \vect{x} + \frac{1}{2} \vect{y}) \in M.
$$
即证明了{\color{red} vector addition}.
\end{proof}

\begin{definition}
\rm 对于$M \subseteq \thereal^n$及$\vect{a} \in M$. $M$关于$\vect{a}$的{\color{red}平移}(translate)定义为
$$
M + \vect{a} = \Set{\vect{x}+\vect{a}}{\vect{x} \in M}.
$$
\end{definition}

\begin{definition}
\rm 给定仿射集$M$和仿射集$L$,若可以找到一个$\vect{a}$使得满足关系
$$
M = L + \vect{a},
$$
则称$M$和$L$ {\color{red}平行}.
\end{definition}

\begin{proposition}
\rm 仿射集的平移仍为仿射集.
\end{proposition}

\begin{proof}
给定仿射集$M$和它的平移$M + \vect{a}$,取任意的$\vect{x},\vect{y} \in M$,那么
$$
(1-\lambda)(\vect{x} + \vect{a})+\lambda(\vect{y}+\vect{a}) = (1-\lambda)\vect{x} + \lambda\vect{y} + \vect{a} \in  M + \vect{a}.
$$
\end{proof}

\begin{proposition}
\rm 仿射集的平行是一种{\color{red}等价关系}.
\end{proposition}

\begin{proof}
自反性,取$\vect{a} \in M$即得;对称性,若$M = L + \vect{a}$,则$L = M + (-\vect{a})$;传递性, 若$M = L + \vect{a},L = L_1 + \vect{b}$,则$M = L_1 + (\vect{a}+\vect{b})$.
\end{proof}

\begin{theorem}
\rm 每个非空仿射集$M$一定平行于唯一的子空间$L$.这个$L$由
$$
L = M-M = \Set{\vect{x}-\vect{y}}{\vect{x},\vect{y} \in M}.
$$
给定.
\end{theorem}

\begin{proof}
先证{\color{blue}唯一},假设$M$同时平行于两个子空间$L_1,L_2$,我们前面已经证明平行是一个等价关系,因此存在$\vect{a}$使得$L_1 = L_2 + \vect{a}$,由于$\vect{0} \in L_1$,所以$\vect{-a} \in L_2$,所以$\vect{a} \in L_2$, 那么$L_1 = L_2$(相当于往子空间里面在扔一个它自己的元素进去,并不会改变这个子空间). 

现在我们要{\color{blue}找到这样一个子空间},根据前面的定理知道含$\vert{0}$的仿射集才是子空间,很明显使得$\vect{a} \in M$,那么$\vect{0} \in M-\vect{a}$. 这样的$\vect{a}$是任意的,并且所有仿射$M+\vect{a}$都是相同的,所以这样所以$L =\sum_{a \in M} M-a = M-M$.
\end{proof}

\begin{definition}
\rm 与非空仿射集平行的子空间的维数称为这个仿射集的{\color{red}维数}. 维数为$0$,$1$和$2$分别称为点,线,面. $\thereal^n$中$(n-1)$维仿射集被称为{\color{red}超平面}(hyperplane).
\end{definition}

\begin{definition}
\rm 给定$\thereal^n$上子空间$L$,对于任意$\vect{y} \in L$,使得关系
$$
\inprod{\vect{x}}{\vect{y}} = 0
$$
成立的$x \in \thereal^n$构成的集合称为$L$的{\color{red}正交补}(orthogonal complement),记为$L^\perp$. 显然$L^\perp$也是一个子空间,且满足
$$
\dim L + \dim L^\perp = n.
$$
\end{definition}

\begin{definition}
\rm {\color{red} (建立在正交性理论上平行于超平面仿射集表示方法)} 若一维子空间是有当个单个非零向量$\vect{b}$构成,那么其对应的正交$n-1$维子空间可以表示为$\Set{\vect{x} \in \thereal^n}{\vect{x} \perp \vect{b}}$. 自然地,与其平行的仿射集为
$$
\begin{array}{ll}
\Set{\vect{x} \in \thereal^n}{\vect{x} \perp \vect{b}} + \vect{a} &= \Set{\vect{x}+\vect{a}}{\inprod{\vect{x}}{\vect{b}}=0~\text{and}~\vect{x} \in \thereal^{n}} \\
&= \Set{\vect{y} \in \thereal^{n}}{\inprod{\vect{y}-\vect{a}}{\vect{b}}=0} \\
&= \Set{\vect{y} \in \thereal^{n}}{\inprod{\vect{y}}{\vect{b}}=\beta} 
\end{array},
$$
其中$\beta = \inprod{\vect{a}}{\vect{b}}$.
\end{definition}

\begin{theorem}
\rm 给定$\beta \in \thereal$以及非零$\vect{b} \in \thereal$,集合
$$
H = \Set{\vect{x} \in \thereal^n}{\inprod{\vect{x}}{\vect{b}}=\beta}
$$
为$\thereal^n$的超平面. 
\end{theorem}

\begin{proof}
研究一下同一$\beta$,得到的$\vect{a}$的关系.
\end{proof}


\end{document}