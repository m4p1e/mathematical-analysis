\documentclass{article}

\usepackage{ctex}
\usepackage{tikz}
\usetikzlibrary{calc}
\usetikzlibrary{cd}
\usetikzlibrary{decorations.pathreplacing}

\usepackage{amsthm}
\usepackage{amsmath}
\usepackage{amssymb}

\usepackage{hyperref} %url
\hypersetup{
    colorlinks=true,
    linkcolor=blue,
    filecolor=magenta,      
    urlcolor=cyan,
    pdftitle={Overleaf Example},
    pdfpagemode=FullScreen,
    }

\usepackage{enumitem}

\usepackage[textwidth=18cm]{geometry} % 设置页宽=18

\usepackage{blindtext}
\usepackage{bm}
\parindent=0pt
\setlength{\parindent}{2em} 
\usepackage{indentfirst}



\usepackage{xcolor}
\usepackage{titlesec}
\titleformat{\section}[block]{\color{blue}\Large\bfseries\filcenter}{}{1em}{}
\titleformat{\subsection}[hang]{\color{red}\Large\bfseries}{}{0em}{}
%\setcounter{secnumdepth}{1} %section 序号

\newtheorem{theorem}{Theorem}[section]
\newtheorem{lemma}[theorem]{Lemma}
\newtheorem{corollary}[theorem]{Corollary}
\newtheorem{proposition}[theorem]{Proposition}
\newtheorem{example}[theorem]{Example}
\newtheorem{definition}[theorem]{Definition}
\newtheorem{remark}[theorem]{Remark}
\newtheorem{exercise}{Exercise}[section]
\newtheorem{annotation}[theorem]{Annotation}

\newcommand*{\xfunc}[4]{{#2}\colon{#3}{#1}{#4}}
\newcommand*{\func}[3]{\xfunc{\to}{#1}{#2}{#3}}

\newcommand\Set[2]{\{\,#1\mid#2\,\}} %集合
\newcommand\SET[2]{\Set{#1}{\text{#2}}} %

\newcommand{\norm}[1]{\left\lVert#1\right\rVert} % 范数
\newcommand{\vect}[1]{\mathbf{#1}} % vector

\newcommand{\thereal}{\mathbf{R}} %实数 

\begin{document}
\title{凸函数的世界 \\ \small{{\color{red}凸分析}和{\color{red}凸优化}}}

\author{枫聆}
\maketitle
\tableofcontents

\newpage
\section{数学优化问题}

%the new world!
\begin{definition}
\rm {\color{red}数学优化}问题或者说{\color{red}优化问题}可以写成如下形式
$$
\begin{array}{ll}
\text{minimize} & f_0(x) \\
\text{subject to} & f_i(x) \leq b_i, \; i = 1,2,\cdots,m.
\end{array}
$$
其中向量$x = (x_1,\cdots,x_n) \in \mathbf{R}^n$称为问题的{\color{red}优化变量},函数$\func{f_0}{\mathbf{R}^n}{\mathbb{R}}$称为{\color{red}目标函数},函数$\func{f_i}{\mathbf{R}^n}{\mathbb{R}}$被称为{\color{red}约束函数},常数$b_i$称为{\color{red}约束上限}或者{\color{red}约束边界}.
\end{definition}

\begin{definition}
\rm  那些满足约束的向量$z$,即使得上述不等式成立的向量,它们构成一个{\color{red}解集} $Z$. 这个解集中使得$f_0(z)$最小的那些$x^*$称为当前优化问题的{\color{red}最优解},即
$$
\forall z \in Z, f_i(z) \leq b_i,\;i=1,2,\cdots,m~\text{and}~f(x^*) \leq f(z).
$$
\end{definition}

\newpage
\section{基本概念}

\subsection{仿射集(affine set)}

\begin{definition}
\rm {\color{red} ($\thereal^n$上直线的定义)} 对任意两个$\thereal^n$中不同两个$\mathbf{x}$和$\mathbf{y}$,形如
$$
\mathbf{x} + \lambda(\mathbf{y}-\mathbf{x}) = (1-\lambda)\mathbf{x} + \lambda\mathbf{y},\; \lambda \in \thereal  
$$
的点集被称为过$\mathbf{x}$和$\mathbf{y}$的直线.
\end{definition}

\begin{definition}
\rm 对于$\thereal^n$中的子集$M$,如果对于任意的$\vect{x},\vect{y} \in M$和$\lambda \in \thereal$都有$(1-\lambda)\vect{x} + \lambda\vect{y} \in M$,则称$M$为$\thereal^n$中的{\color{red}仿射集}(affine set). 相关书与仿射集同义的名词有{\color{red}仿射流形
}(affine manifold),{\color{red}仿射变量}(affine variety),{\color{red}线性变量}(linear variety)或者{\color{red}flat}(平坦的).
\end{definition}


\end{document}